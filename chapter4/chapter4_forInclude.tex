
\setcounter{chapter}{3}
\chapter{Bayes' rule and its applications}

\section{The chain rule}

This chapter is going to focus on how to re-write joint and conditional probabilities. When we turn to statistics later on, it will
turn out that it is often hard to define a joint distribution over many variables. Likewise, it can be hard to calculate 
the probability distribution of a RV $ X $ conditioned on a RV $ Y $ but it may be much easier to find the distribution of $ Y $
conditioned on $ X $. In this chapter we are essentially trying to find simpler expressions for distributions that may be hard to
compute.

The first general method for simplifying a joint distribution is known as the \textbf{chain rule}. \chris{skip? We could actually have introduced
the chain rule much earlier, when talking about the probability of events. However, we chose not to do so as it would have 
cluttered the presentation. Moreover, we only need the chain rule now.} For completeness' sake, we are going to formulate the chain rule first for events and then for random variables.

\begin{Theorem}{\textbf{(Chain rule)}} \label{thm:chain}
The joint probability of events $ E_{1}, \ldots, E_{n} $ can be factorised as
$$ \mathbb{P}(E_{1}, \ldots, E_{n}) = \mathbb{P}(E_{1}) \times \mathbb{P}(E_{2}|E_{1}) \times \ldots \times \mathbb{P}(E_{n}|E_{1}, \ldots, E_{n-1}) $$
\end{Theorem} 
Recall from Definition~\ref{def:jointprob} the notation $\mathbb{P}(E_1,E_2) = \mathbb{P}(E_1 \cap E_2)$ for denoting the probability that both events $E_1$ and $E_2$ occur. There are a couple of things to note about the chain rule: First of all, the numbering of the events is arbitrary. That means that it does not matter in which
order we decompose the joint probability. We could just as well start with any $ E_{i} $ for $ 1 \leq i \leq n $. Second we used the 
word \textit{factorise}. This simply means that we decompose any expression (in this case a joint probability) into a product. Products are
nice in that we can arrange them in any order that we like (i.e.\ they commute). Moreover, products make a lot of calculations easier, as we will
see later.

Let us go ahead and actually prove the chain rule. 
\paragraph{Proof of Theorem~\ref{thm:chain}} We are going to do so inductively and choose $ \mathbb{P}(E_{1}, E_{2}) $ as our
base case. Then we simply employ the definition of conditional probability to get
\begin{equation}
\mathbb{P}(E_{1}, E_{2}) = \mathbb{P}(E_{1}) \times \dfrac{\mathbb{P}(E_{1}, E_{2})}{\mathbb{P}(E_{1})} = \mathbb{P}(E_{1}) \times \mathbb{P}(E_{2}|E_{1})
\end{equation}

Let us assume that the chain rule holds for events $ E_{1}, \ldots, E_{n-1} $. We will abbreviate them as $ E_{1}^{n-1} $. Then we get
\begin{equation}
\mathbb{P}(E_{1}^{n-1}, E_{n}) = \mathbb{P}(E_{1}^{n-1}) \times \dfrac{\mathbb{P}(E_{1}^{n-1}, E_{n})}{\mathbb{P}(E_{1}^{n-1})} 
= \mathbb{P}(E_{1}^{n-1}) \times \mathbb{P}(E_{n}|E_{1}^{n-1})
\end{equation}

Since $ \mathbb{P}(E_{1}^{n-1}) $ factorises according to the chain
rule by our induction hypothesis, we have completed the proof.
$ \square $\bigskip

The chain rule can make our lives even simpler if we have independent events. Assume we want to compute the joint probability of 3 events 
$ E_{1},E_{2},E_{3} $ and we also know that $ E_{1} \bot E_{2} $. In this case our factorisation becomes \eqref{simpleFactor} where
the first equality follows from the chain rule and the second equality follows from independence between $ E_{1} $ and $ E_{2} $.
\begin{align} \label{simpleFactor}
\mathbb{P}(E_{1}, E_{2}, E_{3}) &= \mathbb{P}(E_{1}) \times \mathbb{P}(E_{2}|E_{1}) \times \mathbb{P}(E_{3}|E_{1},E_{2}) \\
&= \mathbb{P}(E_{1}) \times \mathbb{P}(E_{2}) \times \mathbb{P}(E_{3}|E_{1},E_{2}) \nonumber
\end{align}

We can now state the chain rule for random variables. There are two ways you can go about proving it. Either you 
calculate the probability of a specific setting of the variables or you just do the proof based on the distributions of the RVs.
So in the first case you would have to prove that
\begin{align*}
\forall x_1,\ldots,x_n: &P(X_{1} = x_{1}, \ldots, X_{n} = x_{n}) \\
&= P(X_{1} = x_{1}) \times \ldots \times P(X_{n} = x_{n}|X_{1}=x_{1}, \ldots, X_{n-1} = x_{n-1})
\end{align*}
whereas in the second case you would simply prove that
\begin{align*}
P_{X_{1}^{n}} = \overset{n}{\underset{i=1}{\sum}}P_{X_{i}|X_{1}^{i-1}}
\end{align*}

Incidentally, we also introduce a very short notation for the chain rule above. Note that it is not quite correct, since if
$ i = 1 $ we would be conditioning on $ X_{0} $. That is not to bad however, since we can always define ourselves a constant variable $ X_{0} $ that does not affect the distribution. Moreover, this notation is really just meant to be convenient, so you should just accept it as is when you encounter it in papers.

\begin{Exercise}
Prove the chain rule for random variables. The proof is totally analogous to the one give for events.
\end{Exercise}

\begin{Exercise}
Let $X_0$ be a constant RV, i.e.\ there exists $c \in \mathbb{R}$ such that $P(X_0 = c)=1$. 
Prove that $X_0$ is independent of any set of other random variables $X_1,\ldots,X_n$.
\end{Exercise}

\section{Bayes' rule}

In this section we are going to prove \textbf{Bayes' rule}. The rule follows directly from the chain rule.
The proof is really simple and thus of no great interest in and by itself. The consequences of Bayes' rule
are huge however. It will basically allow us to invert a conditional probability distribution. You may rightfully
ask: what's the deal? Well, as we said in the beginning, it may be hard to compute a conditional distribution in one
direction but much easier to compute it in the other direction. On top of that, Bayes' rule opens up a whole range of new possibilities. We will discuss those as we proceed in this chapter.
\begin{Theorem}{\textbf{(Bayes' rule)}}
The probability distribution of a random variable $ X $ given a random variable $ Y $ can be computed as
$$ P_{X|Y} = \dfrac{P_{Y|X}P_{X}}{P_{Y}} $$
\end{Theorem}

And here comes the proof:
\begin{equation}
P_{X|Y} = \dfrac{P_{XY}}{P_{Y}} = \dfrac{P_{Y|X}P_{X}}{P_{Y}} \, . \qquad  \square
\end{equation}

That was the proof! Considering how simple it was, it will be surprising to see what kind of benefits we can get out
of Bayes' rule. To get us started, let us introduce some terminology. In particular, each of the terms
in Bayes' rule has a specific name. You should really learn these names by heart as they crop up all over the place.

$$ \mathit{posterior} = \dfrac{\mathit{likelihood} \times \mathit{prior}}{\mathit{marginal~likelihood}} $$

The posterior is what we get after we have completed the computation. However, its name is related to the prior.
The prior is just the probability that we would place on $ P(X=x) $ \textit{a priori}. Therefore $ P_{X} $
is also known as the prior distribution. When we multiply the prior by the likelihood we get a new distribution
over $ X $ that is conditioned on $ Y $ \chris{after multiplication, we get the joint distribution of X and Y??}. This is the distribution that we place on $ X $ \textit{a posteriori}, i.e.
after having taken into account information about $ X $ that we may get from knowing the value of $ Y $. The marginal
likelihood of $ Y $ is simply needed to normalize the expression to a probability distribution (i.e. to make sure that
it sums to one). Why is it called marginal likelihood? The reason for this is how you can compute it. Recall that when
we are given a joint distribution $ P_{XY} $, we can obtain the distribution
$ P_{Y} $ by simply marginalizing over $ X $.
\begin{equation}
P(Y=y) = \sum_{x \in \supp(X)}P(X=x, Y=y)
\end{equation}

In addition to that, the chain rule allows us to factorise the joint probability. Thus we get
\begin{equation}
P(Y=y) = \sum_{x \in \supp(X)} P(Y=y|X=x) \times P(X=x)
\end{equation}

If you think that this looks an awful lot like the enumerator of Bayes' rule then you are exactly on the right track.
Essentially, we are just summing over all possible denominators (with respect to $ X $). Let us make this more
concrete with an example. Assume that we are given two coins. One of them is fair, meaning that it is equally probable
to come up heads or tails. The other coin is biased towards tails and we happen to know that its probability to come up
heads is only $ 0.3 $. Which coin is flipped is captured by a random variable $ X $ that takes on the value 0 if the
fair coin is used and the value 1 if the biased coin is used. We have no idea which coin is going to be tossed, it could
be either one. Therefore we set our prior to $ P(X=0) = P(X=1) = 0.5 $.

We flip the chosen coin 10 times and obtain 8 heads. The number of heads obtained during the 10 tosses is going
to be encoded by $ Y $. Since all tosses are independent of each other, $ Y $ will
follow a binomial distribution. For each of the two coins we also know the parameter of the binomial distribution.
For the fair coin it is $ \theta = 0.5 $ and for the biased coin it is $ \theta = 0.3 $. Let us compute each of the
enumerators separately.
\begin{align}
P(Y=8|X=0) \times P(X=0) &= \binom{10}{8} 0.5^8 (1-0.5)^2 \times 0.5 = 0.02195 \label{bayes1}\\
P(Y=8|X=1) \times P(X=1) &= \binom{10}{8} 0.3^8 (1-0.3)^2 \times 0.5 = 0.0007 \label{bayes2}
\end{align}
Remember that $ Y \sim binom(10,\theta) $ and that $ \theta=0.5 $ if $ X=0 $ and $ \theta=0.3 $ if $ X=1 $. 

All that is left do is to compute the marginal likelihood of $ Y $. Luckily for us, $ X $ only assumes two
values, so we only need to add up \eqref{bayes1} and \eqref{bayes2}.
\begin{align}
P(Y=8) = &P(Y=8|X=0) \times P(X=0) \\
&+ P(Y=8|X=1) \times P(X=1) = 0.02265 \nonumber
\end{align}

And finally we can apply Bayes' rule to compute the posterior probabilities of $ X $.
\begin{align}
P(X=0|Y=8) &= \dfrac{P(Y=8|X=0) \times P(X=0)}{P(Y=8)} \\
&= \dfrac{0.2195}{0.02265} = 0.969 \nonumber \\
P(X=1|Y=8) &= \dfrac{P(Y=8|X=1) \times P(X=1)}{P(Y=8)} \\
&= \dfrac{0.0005}{0.02265} = 0.031 \nonumber \\
\end{align}

There is a probability of $ 0.969 $ that the fair coin has been tossed when a sequence with eight heads is
generated and only a probability of $ 0.031 $ that the biased coin was tossed. Obviously, the probability of the fair 
coin is much higher. But how much higher? We can take the ratio of the two probabilities. This gives us 
$ \nicefrac{0.969}{0.031} \approx 31 $. We can conclude that the fair coin is 31 times more likely to have generated the sequence with
8 heads than the biased coin. But wait a second, can we maybe find this ratio somewhere else? It turns out that 
the ratio of the likelihoods is the same! That is $ \nicefrac{0.0439}{0.0014} \approx 31 $.

We started out by assuming that both coins were equally likely to be used. However, we then observed a sequence of 10 tosses, 8 of 
which were heads and that made it 31 times more likely that the fair coin was used. What if the priors had not been equal?
Actually, there is a more general story: While calculating the actual probabilities involves a lot of number crunching, just telling whether or not an observation will make one or the other event more likely is not too hard. [For the rest of this chapter, we assume that we only condition on events with non-zero probabilities such as $P(Y=y)>0$ so that we are never dividing by 0].
\begin{align*}
\frac{P(X=x_{1}|Y=y)}{P(X=x_{2}|Y=y)} &= \frac{\dfrac{P(Y=y|X=x_{1})P(X=x_{1})}{P(Y=y)}}{\dfrac{P(Y=y|X=x_{2})P(X=x_{2})}{P(Y=y)}} \\[1em]
&= \frac{P(Y=y|X=x_{1})P(X=x_{1})}{P(Y=y|X=x_{2})P(X=x_{2})}
\end{align*}

From the above equalities, we see that the ratio of the posterior probabilities is determined by the ratio of the likelihood times the
prior. In our coin example, the priors were the same so it was only the likelihood that mattered. If the ratio of any of the above
terms is greater than 1, the posterior will change in favour of $ X=x_{1} $. If the ratio is smaller than 1 the posterior changes
in favour of $ X=x_{2} $. If the ratio is exactly 1, nothing happens\chris{what do you mean by ``nothing happens''???}. 

Notice that in general, although our observations may shift the posterior in favour of $ X=x_{2} $, say, this shift does not necessarily imply that 
$ P(X=x_{2}|Y=y) $ will be greater than $ P(X=x_{1}|Y=y) $. The condition that $ P(X=x_{2}|Y=y) $ is bigger than  $ P(X=x_{1}|Y=y) $ can be rewritten as follows
\begin{align*}
P(X=x_{1}|Y=y) &< P(X=x_{2}|Y=y)  &\Leftrightarrow \\
\dfrac{P(Y=y|X=x_{1})P(X=x_{1})}{P(Y=y)} &< \dfrac{P(Y=y|X=x_{2})P(X=x_{2})}{P(Y=y)} &\Leftrightarrow \\
P(Y=y|X=x_{1})P(X=x_{1}) &< P(Y=y|X=x_{2})P(X=x_{2}) &\Leftrightarrow \\
\dfrac{P(Y=y|X=x_{1})}{P(Y=y|X=x_{2})} &< \dfrac{P(X=x_{2})}{P(X=x_{1})}
\end{align*} 

The last line is of particular interest as it elucidates the relationship between the prior and the likelihood. Only if the likelihood
ratio for $ X=x_{1} $ over $ X=x_{2} $ is smaller than the reversed prior ratio will the posterior probability of $ X=x_{2} $
be greater than that of $ X=x_{1} $. This means that if we have strong (very asymmetric) priors (like $ P(X=x_{1}) = 0.9 $
and $ P(X=x_{2}) = 0.1 $), the likelihood needs to discriminate very well between the two cases in order to tip the scale in
favour of $ X=x_{2} $. In that sense the prior and the likelihood can be seen as battling forces whose equilibrium gives us 
the posterior.

But enough theory about Bayes' rule, it is about time you apply it! To that end, we present you an exercise that is, in some variation,
contained in virtually every textbook on probability theory, statistics or machine learning. Have fun with it!

\begin{Exercise}
A random person walks into the doctor's office to be tested for a particular disease. The disease can be fatal if not treated. However,
successful treatment is possible if the disease is discovered early enough. It is commonly known that the disease occurs in 1 out
of 1000 people of the country's population. The doctor will administer a test that with a probability of 99\% returns a positive results
if the patient does indeed have the disease. At the same time, the test also returns a positive result in 5\% of the cases where the
patient does not have the disease. After the test has been administered to the patient in question, it returns a positive result.
What is the probability that the patient is infected with the disease? 
\\
Proceed as follows:
\begin{enumerate}
\item Write down a guess for what you think the probability might be (do not consider any math at this point).
\item Calculate that probability.
\item Check whether there is a considerable difference between your initial guess and the calculated probability. Go on to examine
how the different factors have influenced the probability of the patient having the disease.
\end{enumerate}
\end{Exercise}

Let us finish up this section with some more notation. In many applications of Bayes' rule we only want to know which outcome is
the most likely, without worrying too much about the actual probabilities. Likewise, there is a range of situations where we
just want to assign a score to outcomes and do not demand this score to be a probability. Throughout this chapter,
we have repeatedly encountered the following phenomenon: In order to rank the values of an RV according to their probabilities, we do not necessarily need to compute the marginal likelihood since it cancels in all these comparisons anyway. Therefore, you will often see authors stating that
\begin{equation} \label{proportionality}
P(X=x|Y=y) \propto P(Y=y|X=x)P(X=x)
\end{equation}

This equation reads as ``the posterior is proportional to the product of the likelihood and the prior''. In general, if we have two quantities
$ a $ and $ b $, then by $ a \propto b $ we mean that there is some constant $ C \in \mathbb{R} $ such that $ a = cB $. Notice
that the probability distribution is a function and hence we require $ C $ to be the same across the domain of that function (that
is $ C $ should be the same for all values of $ X $).

\begin{Exercise}
What is the value of $ C $ in Equation~\eqref{proportionality}?
\end{Exercise}



\section{Na\"ive Bayes}
In this section, we introduce a rather crude application of Bayes's rule which is surprisingly successful nonetheless.
Assume that instead of one random variable we are observing a sequence of random variables. Thus our problem is the following:
\begin{equation}
P(Y=y|X_{1}^{n}=x_{1}^{n}) \propto P(X_{1}^{n}=x_{1}^{n}|Y=y) \times P(Y=y) 
\end{equation}

By the chain rule we can decompose the right-hand side into
\begin{align}
P(Y=y|X_{1}^{n}=x_{1}^{n})
\propto &P(X_{1}=x_{1}|Y=y) \times \ldots \nonumber \\
&\times P(X_{n}=x_{n}|Y=y,X_{1}^{n-1}=x_1^{n-1}) \times P(Y=y) \nonumber
\end{align}

We are now going to introduce the aforementioned crudeness into the model by assuming that all $ X_1,\ldots,X_n$ are conditionally independent given $ Y $. Notice that
this is just an assumption that we are making without justification. In fact, it is very likely wrong. However, it makes our
live much easier because we only have to deal with very simple terms of the form $ P(X_{i}=x_{i}|Y=y) $. Because of the
crudeness of our assumptions, this probabilistic model is known as \textbf{na\"ive Bayes} (sometimes also 
stupid Bayes).

\begin{Definition}
A na\"ive Bayes model is a probabilistic model that assumes
$$ P_{Y|X_{1}^{n}} \propto P_Y P_{X_{1}|Y} P_{X_{2}|Y} \cdots P_{X_{n}|Y} $$
\end{Definition}
Once we know all the component distributions $P_{X_i|Y}$, calculating the result is pretty straightforward. 

In order to illustrate how na\"ive Bayes works we are going to employ one of its showcase applications where it indeed had
a lot of success in real life. The application we are talking about is text classification. The task is the following: you
are given some documents and for each of the documents you have to assign a label signifying its class. What you consider
a class depends on your actual application setting, but usually classes are broad categories, such as legal texts, medical
texts etc. If you manage to succeed at this task, you can accomplish a lot of things automatically that required humans before. For example, you could tag online news with their relevant categories and people who are interested in
a particular category will then have an easier time finding the news related to that category. Crucially, since you will
write a computer program that does the classification for you, you will not need to read any of the texts yourself. This automation will obviously allow you to classify huge quantities of text in a very short amount of time.

\begin{Exercise}
A collection of text (or any other kind of data for that matter) is often called a \textbf{corpus}. Here we are going to
use a toy corpus. The corpus just consists of two sentences and we assume that each sentence is its own document.
Your task is to classify these two documents correctly and report the posterior probability for the correct label. 
The categories that you can label the documents with are
finance (0), medicine (1) or law (2). You can find the corpus (the pmfs of the distributions) below. For simplicity, we are not going to distinguish between lower and upper case words (this is actually common practice). For better 
readability, we are also using the actual words instead of their numerical encodings as values for the random 
variables. Just remind yourselves that those words are real numbers in reality.
\end{Exercise}

\textbf{The corpus:}
\begin{itemize}
\item evidence ruled irrelevant or inadmissible shall not be considered by the chamber.
\item that is probably what investors who poured \$90 billion in european equity funds this year are hoping for.
\end{itemize}

\section*{Further Reading}
Here, we have only scratched the surface of what Bayes' rule allows us to do. To get a wider outlook on what else is possible,
you can consult \href{http://www.cs.ubc.ca/~murphyk/Bayes/bayesrule.html}{Kevin Murphy's webpage}.

%%% Local Variables:
%%% mode: latex
%%% TeX-master: "chapter4"
%%% End:
