\documentclass[a4paper,11pt,leqno]{report}

\usepackage{amsmath, amssymb, mdframed}
\usepackage{hyperref}
\hypersetup{colorlinks=true, urlcolor=blue, breaklinks=true}

\newmdtheoremenv{Definition}{Definition}
\newmdtheoremenv{Exercise}{Exercise}

\title{Basic Probability}
\date{}

\begin{document}

\chapter{Axiomatic Probability Theory}

\section{Axioms of probability}
In the previous chapter, we have introduced sample spaces and event spaces. 
In a mathematically concrete sense, we are now going to measure the probability of events. 

\begin{Definition}
A measure is a function $ \mu: \mathcal{S} \rightarrow \mathbb{R} : S \mapsto \mu (S) $ that maps elements
from a set of sets $ \mathcal{S} $ (formally a \href{http://en.wikipedia.org/wiki/Sigma-algebra}
{$ \sigma $-algebra}) to real numbers. Such a measure has the following properties:
\begin{enumerate}
\item $ \mu(S) \geq 0 $ for all $ S \in \mathcal{S} $
\item $ \mu\left( \underset{i = 1}{\overset{\infty}{\bigcup}} S_{i} \right)
= \underset{i = 1}{\overset{\infty}{\sum}} \mu \left( S_{i} \right) $
\end{enumerate}
\end{Definition}

The second property ensures that we can compute the measure of the union of two or more sets by
first computing the measure of each set individually and then simply adding them up. We have already
seen an example of a measure, namely the counting measure $ |\cdot| $ that counts the elements in a set.

There is one measure, however, that is going to be the star of the rest of this script, namely 
the \textbf{probability measure}.

\begin{Definition}
A probability measure $ \mathbb{P}: \mathcal{A} \rightarrow \mathbb{R} : A \mapsto \mathbb{P}(A) $
on an event space $ \mathcal{A} $ associated with a sample space $ \Omega $ to the real numbers has the
following properties.
\begin{enumerate}
\item $ \mathbb{P}(A) \geq 0 $ for all $ A \in \mathcal{A} $
\item $ \mathbb{P}\left( \underset{i = 1}{\overset{\infty}{\bigcup}} A_{i} \right)
= \underset{i = 1}{\overset{\infty}{\sum}} \mathbb{P} \left( A_{i} \right) $ \label{union}
\item $ \mathbb{P}(\Omega) = 1 $ \label{unity}
\end{enumerate}
\end{Definition}

Notice that we only added item \ref{unity} to the general definition of a measure. This has some
interesting consequences. For one, since $ \mathbb{P}(\Omega) = 1 $ we know that no event can have
probability greater than $ 1 $. Moreover, we have 
\begin{equation}
\mathbb{P}(\Omega) = \mathbb{P}(\Omega \cup \emptyset) = \mathbb{P}(\Omega) + \mathbb{P}(\emptyset)
\end{equation}
which implies that $ \mathbb{P}(\emptyset) = 0 $ always holds. We can therefore safely exclude 
the empty set from our event spaces.

We have already discussed uniform probability in the previous chapter. Now we can formally explain
what we meant by that. The uniform probability measure is the measure $ \mathbb{P} $ such that
$ \mathbb{P}(\{\omega\}) = \frac{1}{|\Omega|} $ for all $ \omega \in \Omega $. This is where the
distinction between sample and event spaces becomes important. We cannot measure the elements of a
sample space, only the elements of an event space! Recall our convention that we will always assume
that $ \mathcal{A} = \mathcal{P}(\Omega) $ which obviously contains a singleton for each element in
$ \Omega $. This implies the uniform probability measure is indeed well-defined. Whenever we talk about
\textit{uniform probability} we either mean the uniform probability measure or, more often, the real
value $ \frac{1}{|\Omega|} $ to which this measure uniformly evaluates.

In order to create a tight relationship between a sample space, an event space and a probability measure,
we can introduce the concept of a \textbf{probability space}. Probability spaces are also know as 
(probabilistic) \textbf{experiments}.

\begin{Definition} \label{ProbabiltySpace}
A probability space is a triple $ (\Omega, \mathcal{A}, \mathbb{P}) $, consisting of a sample space $ \Omega $,
an event space $ \mathcal{A} $ and a probability measure $ \mathbb{P} $.
\end{Definition}

If we want to roll a die, for example, we have the sample space $ \Omega = \{1,2,3,4,5,6\} $ and, by convention,
the event space $ \mathcal{A} = \mathcal{P}(\Omega) $. If we add the uniform probability measure, we have
constructed ourselves an experiment. We can use it to answer a couple of questions. For example, we might ask 
what is the probability of obtaining an even number. By item \ref{union} of our definition, this probability
is given as $ \mathbb{P}(\{2,4,6)\} = \mathbb{P}(\{2\} \cup \{4\} \cup \{6\}) = \mathbb{P}(\{2\}) + \mathbb{P}(\{4\})
+ \mathbb{P}(\{6\}) = \frac{1}{6} + \frac{1}{6} + \frac{1}{6} = \frac{1}{2} $.

It is important to point out that we just chose the uniform probability measure as the one that seems ``natural'' for
a die roll. However, nobody is forcing us to do so. In fact, def. \ref{ProbabiltySpace} allows us to impose arbitrary
probability measures.

\begin{Exercise}
Take $ (\Omega, \mathcal{A}, \mathbb{P}) $ with $ \Omega $ and $ \mathcal{A} $ as before but now use the 
probability measure $ \mathbb{P} = \{(\{1\},0), (\{2\}, \frac{1}{12}), (\{3\}, \frac{1}{6}), (\{4\}, \frac{1}{6}), (\{5\}, \frac{1}{3}),
(\{6\},\frac{1}{4}) \} $.
\begin{enumerate}
\item Verify that $ \mathbb{P} $ is indeed a probability measure.
\item Compute the probability of obtaining a number smaller than $ 5 $ in this experiment.
\end{enumerate}
\end{Exercise}

\end{document}